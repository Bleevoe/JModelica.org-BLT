\documentclass[11pt,a4paper]{article}

\usepackage[pdfpagelabels]{hyperref}
\hypersetup{colorlinks = true, linkcolor = black, citecolor = black, urlcolor = blue}

\usepackage{graphicx}
\usepackage[utf8]{inputenc}
\usepackage[T1]{fontenc}
\usepackage{mathptmx}
\usepackage{parskip}
\usepackage{url}
\usepackage{booktabs}
\usepackage{multirow}
\usepackage[english]{babel}
\usepackage{amsmath,amssymb}
\usepackage{float}
\usepackage[linesnumbered]{algorithm2e}
\widowpenalty=5000
\clubpenalty=5000

\allowdisplaybreaks

\newcommand*\xbar[1]{%
  \hbox{%
    \vbox{%
      \hrule height 0.5pt % The actual bar
      \kern0.5ex%         % Distance between bar and symbol
      \hbox{%
        \kern-0.1em%      % Shortening on the left side
        \ensuremath{#1}%
        \kern-0.1em%      % Shortening on the right side
      }%
    }%
  }%
}

\usepackage{geometry}            %% please don't change geometry settings!
%\geometry{left=20mm, right=20mm, top=25mm, bottom=25mm, noheadfoot}
\geometry{left=30mm, right=30mm}


\begin{document}

\begin{equation} 
\label{eq:cdc_ST}
\begin{aligned}
\dot{v}^X - v^Y \dot{\psi}  &= \frac{1}{m} ( F^x_{f}\cos(\delta) +
F^x_{r} -F^y_{f}\sin(\delta) ),  \\
\dot{v}^Y + v^X \dot{\psi}  &= \frac{1}{m} ( F^y_{f}\cos(\delta) +
F^y_{r} +F^x_{f}\sin(\delta) ), \\
I_{ZZ} \ddot{\psi}  &= l_f F^y_{f}\cos(\delta) - l_r F^y_{r} + l_fF^x_{f}\sin(\delta),
\end{aligned}
\end{equation}
where $m$ is the vehicle mass, $I_{ZZ}$ is the vehicle inertia
about the $Z$-axis,
$\dot{\psi}$ is the yaw rate, $\delta$ is the steer angle,
$[v^X,v^Y]$ are the longitudinal and lateral velocities at the
center of gravity, $[l_{f}$, $l_{r}]$ are the distances from the mass center
to the front and rear wheel base, and $[F^{x}, F^{y}]$ are the longitudinal and
lateral tire forces acting at the front and rear wheels.


Fig.~\ref{fig:dt} provides a schematic of the double-track model that is used in the low-level formulation. It has five degrees of freedom: two translational ($v^X$ and $v^Y$) and three rotational (the roll-pitch-yaw angles $(\phi,\theta,\psi)$). The suspension  model is a rotational spring-damper system, and longitudinal and lateral load transfer is included. The derivation and details of both models are found in \cite{berntorp2014phd,berntorp_model}.

The nominal tire forces $F^x_{0}$ and $F^y_{0}$ for the longitudinal and lateral directions under pure slip conditions are computed with the Magic formula \cite{Pacejka2006}, given by
\begin{equation}
	\label{eq:Fx}
	\begin{aligned}
		F^x_{0} &= \mu_x F^z 
		\sin\Big( C_{x} 
		\arctan\big(B_{x} \lambda_i \\
		& \quad -E_{x}(B_{x}\lambda-\arctan B_{x}\lambda)\big)\Big),
		\\ 
		F^y_{0} &= \mu_y F^z \sin\Big( C_{y} \arctan\big(B_{y} \alpha \\
		& \quad  -E_{y}(B_{y}\alpha-\arctan B_{y}\alpha)\big)\Big),
	\end{aligned}
\end{equation}
with  lateral slip $\alpha_i$ and  longitudinal slip
$\lambda_i$ 
defined as
\begin{subequations}
	\begin{align}
		\dot{\alpha}_i \frac{\sigma}{v^x_{i}} + \alpha_i &:= -\arctan \left( \frac{v^y_{i}}{v^x_{i}} \right), \label{eq:alpha}\\
		%\lambda_f &= \frac{R_e \omega_f - v_{x,f}}{v_{x,f}}, \label{eq:lambdaf} \\
		%\lambda_r &= \frac{R_e \omega_r - v_{x,r}}{v_{x,r}}, \label{eq:lambdar} \\
		%v_{x,f} &= v_x \cos(\delta) + (v_y + l_f \dot{\psi}) \sin(\delta), \\
		%v_{x,r} &= v_x,\label{eq:vw}
		\lambda_i &:= \frac{R_w \omega_i - v^x_{i}}{v^x_{i}},\label{eq:lambda} 
	\end{align}\label{eq:slip}%
\end{subequations}
where $\sigma$ is the relaxation length, $R_w$ is the wheel radius,
$\omega_{i}$ is the wheel angular velocity for wheel ${i\in \{f,r\}\ \text{or} \ \{1,2,3,4\}}$, and
$[v^y_{i},v^x_{i}]$ are the lateral and longitudinal wheel
velocities for wheel $i$. In the following we suppress the index $i$ for brevity. 
In
\eqref{eq:Fx}, $\mu_x$ and $\mu_y$ are friction
coefficients and $B$, $C$, and $E$ are parameters.
 The nominal 
normal force acting on each wheel axle is given by
\begin{equation*} %\label{eq:Fz0}
F^z_{0,f} =  mg\frac{l_r}{l}, \quad F^z_{0,r} =  mg\frac{l_f}{l}, 
\end{equation*}
where $g$ is the gravitational acceleration and $l = l_f + l_r$. In the single-track model $F^z = F^z_0 $ in \eqref{eq:Fx}. This is not true for the double-track model, because of load transfer.

An experimentally verified approach to tire modeling under combined slip constraints is to scale the
nominal forces \eqref{eq:Fx} with a weighting
function $G$ for each direction, which depends on $\alpha$ and $\lambda$ \cite{Pacejka2006}. The relations are
\begin{equation}
\label{eq:cdc_WFX}
\begin{aligned}
F^{x,y} &= F^{x,y}_{0} G_{m},  \\
G_{m} &= \cos( C_{m} \arctan(H_{m} m) ) , \\
H_{m} &= B_{m 1} \cos(\arctan(B_{m 2} m)),
\end{aligned}
\end{equation}
where $m$ is either  $\alpha$ or $\lambda$.
Moreover, since it is the torques that can be controlled in a physical setup, we introduce a model for the wheel dynamics, namely
\begin{equation*}
\tau = I_w \dot{\omega}-R_wF^x, %\quad \label{eq:gvmwd}
\end{equation*}
where $I_w$ is the wheel  inertia and $\tau$ is the input torque.
To account for that  commanded steer angle and brake/drive torques are not achieved instantenously, we incorporate first-order models from reference to achieved value according to
\begin{equation}
T\dot \delta = -\delta+ \delta_\mathrm{ref},\label{eq:cdc_inpmodel}
\end{equation} 
and similarly for the torques, where $T$ in \eqref{eq:cdc_inpmodel} is the time constant of the  control loop. The parameter values used here correspond to a medium-sized passenger car on dry asphalt.

\end{document}